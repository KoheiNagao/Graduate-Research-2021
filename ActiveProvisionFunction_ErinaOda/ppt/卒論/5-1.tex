%第5-1章:評価

\section{評価}

第3章では,感染症予防サポートシステムは,感染症予防の観点から感染リスクのレベルを通知するとともに,感染リスクを軽減する環境づくりをサポートするという目的を基にして,下記の2点の要求事項を満たす必要があるとした.

\begin{itemize}
	\item 室内環境が測定できること.
	\item 設定した感染リスクの基準に従って通知ができること.
\end{itemize}

「室内環境が測定できること」という要求事項に関して,作成したシステムは二酸化炭素濃度,温湿度,室内滞在人数の測定ができるため,要求を満たすことができたといえる.
「設定した感染リスクの基準に従って通知ができること」という要求事項に関しては,警戒レベル,換気要請の基準,入室危険度を設定し,警戒レベルおよび入室危険度はLED,換気要請はブザーによって,ユーザーに感染リスクの情報を,視覚や聴覚で分かりやすく能動的に通知することができた.室外のユーザーには室外デバイスによるLEDでの入室危険度の通知,室内のユーザーにはブザーとLEDによる換気要請や警戒レベルの通知というように,対象とするユーザーによって提供する情報,提供の方法を変えることで,室内外から共に感染リスクを軽減する環境づくりをサポートするシステムとなった。