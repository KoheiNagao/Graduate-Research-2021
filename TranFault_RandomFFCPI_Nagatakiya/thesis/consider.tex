%考察
本章では,実験結果に対する考察について述べる.

%\section{評価・考察}
マルチサイクルテストにおける遅延故障検出結果として,
キャプチャ数を増やすと故障検出率が低下する傾向にあることが判明した.
この結果は,キャプチャ数を増加させるにつれて内部状態が次第に遷移しなくなるという,
``故障検出能力低下問題''に起因する可能性が高い.
また,s9234回路に関してのみ,キャプチャ数の増加に伴い故障検出率が増加しているが,
これは回路そのものが原因であると推測する.

FF制御を行った場合の実験結果では,
s5378回路では,ほとんどのキャプチャ数において故障検出率が1.5\%程度向上し,
s9234回路では,いくつかのキャプチャ数において故障検出率が0.8\%ほど向上したことが明らかとなった.
しかしながら,s13207回路においては,ほとんどのキャプチャ数において故障検出率が悪化する結果となり,
FF制御が遅延故障検出率の向上に必ずしも寄与するとは言い難い結果となった.
回路ごとに最適なCP挿入箇所が存在するものの,本実験ではCP挿入箇所をランダムに選定したことが,
今回の得られた結果の原因である可能性が高い.
CP挿入箇所選定手法に関して,より適した選定点の提案を実現できれば,
遅延故障検出率を向上させることができる.