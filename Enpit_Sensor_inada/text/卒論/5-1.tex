%第5-1章:評価
%表を作成して,比較評価.評価軸を設定し.表にして比較.定量的評価は数値,定性的評価は〇△×で.


\section{評価}

本システム,および私が実装を担当したセンサデバイスの評価を行う.

評価は,3章で述べた以下の要求が満足しているかを基準に行う.
\begin{enumerate}
    \item あらかじめ指定されたガイドラインだけでなく,部屋の特性にも配慮して評価を行える.
    \item 導入が容易でセンサデバイスを部屋の任意の場所に設置できる.
    \item 在室している人に対して感染リスク,および対処法を分かりやすく示すことができる.
    \item 入室しようとしている人に対しても部屋の滞在人数や感染リスクを分かりやすく示すことができる.
\end{enumerate}
まず,部屋の特性にも配慮しているかという点においては,感染リスクという指標を用いることで,長期的に部屋の特性を反映するものとなっている.
また,その部屋の特性の把握において,換気状態の目安となる二酸化炭素濃度を用いているため,部屋の見取り図だけではわからないような空気の流れ等も反映しているものになっている.
これらのことより,部屋の特性には配慮するという要求は満たせていると考える.
続いて,導入が容易かという面においては,各デバイスが無線でやり取りをするため,従来のものより配線などについても考える必要がなくなったという点で容易なものとなっている.
% また,センサを部屋の任意の場所に設置できるという点においては,今回実装したデバイスは電池格納部を除き7cm×5cmのユニバーサル基盤上に実装できたため,これも満たしていると考えられる.
また,センサデバイスを部屋の任意の場所に設置できるという点においては,今回実装したセンサデバイスは図\ref{dev_pic}のように,電池格納部を除き7cm×5cmのユニバーサル基盤上に実装できたため,これも満たしていると考えられる.
\begin{figure}[htbp]
    \centering
    \includegraphics[width = 15cm]{./picture/device_pic.eps}
    \caption{実装したセンサデバイスの写真}
    \label{dev_pic}
\end{figure}
しかしながら,Webカメラ1台で部屋全体が見える必要があり,またセンサデバイスの電源である電池が比較的高電流,高容量である必要があるなど少し制約のある形となってしまった.
よって,この点においては改善することが必要である.
%次に,感染リスクを必要な人に,分かりやすく表示できるという点においては,室外内で表示デバイスを変える,3色でリスクを表現できるなど,直感的に分かりやすくなっていると感じる.
%また,室内についてはブザーも用いるなど,危険な状態がより分かりやすくなる工夫をすることができ,この点は達成されたと考える.
次に,在室している人に対して感染リスク,および対処法を分かりやすく示すことができるという点においては,3色でリスクを表現でき,直感的に分かりやすいものとなっていると考えられる.
また,ブザーを用いて通知したり,換気や温湿度の上下などLEDの色によって適切な対処法をそれぞれ示すことができていたため,この要求は満たせていると考える.
最後に,入室しようとしている人に対してもリスクを分かりやすく表示することができるという点においては,室外に屋内とは別デバイスを用意し,入室前に感染リスクが確認できるものとなっている.
また,部屋の滞在人数などをもとに,感染リスクを3色で簡潔に表現できるため,この点については満たせていると考える.

以上の評価の結果より,本研究の目的である,乾電池で動作し,かつ無線でセンサのデータを送信する,従来より設置場所の制約の少ない小型の室内環境値計測デバイスを開発することについては,導入の容易さという面で改善の余地があるものの,達成することができた.
% 本システムを評価する.3章で設定した,3点の基本の評価軸より評価する.評価したものを書き表\ref{hyouka}に示す.

% \begin{table}[htb]
% \begin{center}
% \caption{システムの評価}
% \begin{tabular}{|c|c|} \hline
% 基本の評価軸 & 評価 \\ \hline \hline
% 従来のセルフレジよりコストは抑えられるか & 〇 \\
% 既存の中小店でも導入が容易か & △ \\
% 従来のセルフレジより簡単な動作で決済まで行えるか & △\\ \hline
% \end{tabular}
% \label{hyouka}
% \end{center}
% \end{table}

% 表\ref{hyouka}を上から順に説明する.従来のセルフレジよりコストは抑えられるかという評価軸について,2.2節で述べた表\ref{taisho}のスーパーマーケットを対象にして確認をする.Raspberry Piの価格は5,700円程度,各種センサと周辺機器の合計価格は3,500円程度のため,カゴにかかる合計価格は9,200円とする.サーバと周辺機器にかかる価格を約150,000円とする.サーバ1台約150,000円とカゴ90個約828,000円とすると,本システムでかかる価格は約978,000円となり,従来のセルフレジとして2.2節で仮定した登録機1台と精算機7台の合計価格の約5\%程の価格となることが分かった.上記の理由から,従来のセルフレジよりコストを抑えられるとした.

% 次に,既存の中小店でも導入が容易かという評価軸においては,現段階では容易ではない.Raspberry Piや各種センサがしっかりと固定されておらず,誰でも導入ができるわけではない.しかし,これからしっかり固定できるような状況ができれば,既存の買い物カゴに設置できる規模感であるため可能性があるという観点から△とした.

% 次に,従来のセルフレジより簡単な動作で決済まで行えるかという評価軸においては,現時点では,商品のバーコードを読み取らせるために商品を回転させ,バーコードリーダを操作する必要はないが,バーコードをWebカメラに向けて台に置く必要があるため,△とした.