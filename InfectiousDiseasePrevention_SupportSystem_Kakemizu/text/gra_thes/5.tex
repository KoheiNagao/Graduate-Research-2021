%第5章
本章では、本システムに対する評価・考察を行い、今後の課題や将来性についても述べる。

まず3.1節に挙げた、本システムが果たすべき2つの大きな役割に対して評価する。3.1節において、本システムが果たすべき役割について、「感染症予防対策のルールを守ってもらうよう働きかける役割」、「感染症予防対策の基準を定める役割」の2つを挙げた。感染症予防対策のルールを守ってもらうよう働きかける役割については、室内環境に応じた換気要請の発出や、感染リスクのレベルの通知によって、換気や人数調整といった具体的なアクションを促すことが実現できていると考えられる。感染症予防対策の基準を定める役割については、利用者が感染症予防のためにとるべき、換気と部屋に滞在する人数の調整というアクションについて、感染症予防の観点から、部屋を安全な状態に保つため、具体的にその基準を定めることで、利用者自身が感染症予防のためにとるべきアクションを明確にすることが実現できていると考えられる。

また本システムでは、センサデバイスで取得したデータを随時データベースに記録しているほか、室内の滞在人数や警戒レベル、感染リスクといったデータも、センサデバイスからのデータ更新に伴って導き出されていることから、必要に応じてデータを保管しておくことで、システム外部で様々なデータの相関を調べることもできる。そのため、感染症予防対策の基準を定める役割に関しては、応用の余地があると考えられ、例えば以下のような応用の仕方が考えられる。

本システムでは、分析に活かせる多くのデータを導き出せるが、中でも設計の段階から分析に役立てられるデータとして着目していたのは、警戒レベルと感染リスクのデータである。既に述べたように、室内にある程度の人数が滞在していないと警戒レベルの導出は行われない。そのため警戒レベルの推移のデータは、その部屋が警戒レベルを導出できる条件下で利用されているとき、どの程度二酸化炭素濃度が高まりやすいかを確認でき、運用ルール改定の基準にできる。また、警戒レベルと感染リスクのデータをシステム内部で分析し、本システムでは固定的である、二酸化炭素濃度と警戒レベル、警戒レベルと滞在可能人数の関係を、部屋の警戒レベルと感染リスクの変動の仕方に応じて流動的に変化させると、よりその部屋にあった感染症予防対策を講じることが可能になると考えられる。

感染リスクのデータは、換気状況など、部屋の運用の仕方が適切であるかどうかを示しているため、部屋が感染症予防対策上、危険な状態で使用されていないかを確認できる。そのため学校やオフィス、公共施設などでの利用のケースを想定すると、時間帯ごとの感染症予防対策への取り組みの徹底度合いが、エビデンスとして残されることから、管理者側からの適切な指導が行えるほか、各部屋の責任者となる者が、感染症予防対策に、より注意して取り組むことができると考える。

本システムの設計時の着想では、利用環境ごとに異なる、床面積の広さ・空間の広さ、換気のしやすさや窓の位置と数、換気設備の有無、部屋利用者の活動の仕方などに柔軟に対応し、利用環境に合わせた感染症予防対策の基準を定め、利用者に感染症予防対策のルールを守ってもらうよう働きかけられることが本システムの特徴であった。実際に、本システムは4.2節の総合テストでの検証のように、様々なシナリオにあった感染症予防の働きかけが可能となることが考えられる。しかしながら、本システムでの感染症予防対策の基準の決め方では、換気のしやすさや、部屋の床面積の広さのわりに、ものが多く置かれているなどの理由から実際の空間が狭いというような部屋の特性が、そのまま二酸化炭素濃度の上昇の仕方に反映されることを前提としている部分があり、柔軟性に欠けていると考える。理想的な環境における本システムの実用性は確認されたものの、部屋ごとの特性を加味した感染症予防対策の基準を、実際の利用環境において適切に定めるためには、本システムでの室内環境の分析の仕方よりも複雑に、室内環境を分析する必要があるとも考えられ、部屋の特性自体をシステムの分析機能によって導き出すことができると、現在のシステムと比較し、よりその部屋の特性に適合した感染症予防対策の基準の設定を行うことができるため、更なる研究と改良の余地が大いに残されていると考える。


