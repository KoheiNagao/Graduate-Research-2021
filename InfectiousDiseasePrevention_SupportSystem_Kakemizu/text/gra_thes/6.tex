%あとがき
本研究ではセンシング技術、物体検出技術、および複数デバイス間の無線通信という3つの技術とデータ分析の組み合わせによって、感染症予防という、研究・実用化が活発に進められる分野において新たな価値を生み出すことができた。感染症予防に関しては、既に様々な分野で研究・開発がなされているものと思われるが、今回私たちが開発したシステムも、感染症予防の取組を援用するシステムとして貢献できると考える。

また今回の研究では、感染症予防のために活用できるシステムの社会的な必要性が高まり、多くの企業や研究機関により研究・開発が進められている状況下で、感染症予防のために用いるシステムとして、3 密回避に役立てられるという、新型コロナウイルスの世界的な流行以前にはなかった新しい価値を持たせることも、1つの目標として定められた。本研究において、私たちの考える感染症予防のサポートシステムの基本形を提案することができた。今後さらなる研究が進められれば、より高いリアルタイム性と精度を併せ持つモニタリングと、感染症予防対策基準の設定機能における更なる柔軟性を実現でき、より利用しやすいものへと改良が進められることから、拡張性のある研究であると考える。
